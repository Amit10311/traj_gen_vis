 



{\itshape This package is devoted to generate an online chasing trajectory in response to the future movement of a moving target only for a short horizon. It assumes that the future trajectory of target is updated with a given time interval and priori map is given in the form of Octomap}

{\itshape I also wish that my package can provide a test enviroment for comparing different chasing algorithm}

\section*{Overview}

\subsection*{1.\+1 Algorithm}



The paper was accepted to I\+R\+OS 2019 and preprinted in ar\+Xiv. If this package was helpful to your project, it would be grateful if you could cite \href{https://arxiv.org/pdf/1904.03421.pdf}{\tt my paper}.


\begin{DoxyCode}
1 @article\{jeon2019online,
2   title=\{Online Trajectory Generation of a MAV for Chasing a Moving Target in 3D Dense Environments\},
3   author=\{Jeon, Boseong Felipe and Kim, H Jin\},
4   journal=\{arXiv preprint arXiv:1904.03421\},
5   year=\{2019\}
6 \}
\end{DoxyCode}


\subsection*{1.\+2 File Structure}

\href{https://icsl-jeon.github.io/traj_gen_vis}{\tt doxygen} (still many to be added)

\section*{2 Installation}

\subsection*{2.\+1 Dependencies}

\subsubsection*{traj\+\_\+gen (with qpoases)}

The package is trajectory generation library which is used for smooth path generation.

\href{https://github.com/icsl-Jeon/traj_gen}{\tt download here}

\subsubsection*{rotors\+\_\+simulator}

The package is gazebo simulator for M\+AV. This is used for simulation of chasing planner in a virtual M\+AV platform

\href{https://github.com/ethz-asl/rotors_simulator}{\tt download here}

\subsubsection*{octomap}

We use octomap to represent the environment. dynamic\+E\+D\+T3D libraries also should be installed for Euclidean distance transform field(\+E\+D\+F). The visibility score field(\+V\+S\+F) will be computed based on the E\+DF.

\href{http://github.com/OctoMap/octomap}{\tt download here}

\section*{3. Usag}

\subsection*{3.\+0 Common procedure -\/ map and target trajectory}

The chasing algorithm receives 1) prior map (in the form of octomap) and 2) The future trajectory of target in either fully known(\href{#oneshot}{\tt one shot mode}) or partially updated way. The environments provided are still to be updated more. Sorry for my laziness ...

{\itshape 1) Map}

In \char`\"{}worlds\char`\"{} folder, you can find $\ast$.bt files (e.\+g. map3.\+bt which appears in the figures of this page). You can build your own map by following the \href{https://github.com/ethz-asl/rotors_simulator/wiki/Generate-an-octomap-from-your-world}{\tt instruction} of rotors simulator wiki

{\itshape 2) target trajectory}

In \char`\"{}data/\$\{map name\}\char`\"{} folder, you can find path$\ast$.txt files which can be loaded in the gui of the package. You may also generate your own path by traj\+\_\+gen package. For saving and loading a target path, \href{https://github.com/icsl-Jeon/traj_gen}{\tt this page} is referred(traj\+\_\+gen). \label{_without}%
 \subsection*{3.\+1 Simulation without gazebo}

This mode does not require gazebo. It only tests the proposed chasing policy and display the result in Rviz. Please run the following command\+:


\begin{DoxyCode}
1 roslaunch auto\_chaser simulation\_without\_gazebo.launch
\end{DoxyCode}


\subsubsection*{Step 1 \+: Spawning chaser from user selection}

This step should precede the following steps. In rviz tool properties widget, set the topic name of 2D Nav Goal as /chaser\+\_\+init\+\_\+pose. If you don\textquotesingle{}t remember the topic, just click the {\ttfamily set chaser pose button} in gui in the next time \+:). If things done, move on to one of the following two modes \+: one-\/shot or receding horizon method

\subsubsection*{Step 2 -\/ option A \+: One-\/shot mode (offline trajectory computation)}



For this one-\/shot mode, the chaser is assumed to have the full information for the future trajectory of target. Based on the future path, the trajectory will be computed immediately.

\subsubsection*{Step 2 -\/ option B \+: Receding horizon mode (online trajectory computation)}



For the receding horizon method, the chaser does not have full information of the future movement of target. Instead, the future movement of target for only short horizon ($\ast$$\ast$\char`\"{}pred\+\_\+horizon\char`\"{}$\ast$$\ast$ parameter in launch files) with a regular interval ({\bfseries pred\+\_\+horizon} -\/ {\bfseries early\+\_\+end\+\_\+time} . Please check it in launch files!) is fed to chaser and chaser should keep re-\/plan in response to the updates.

\subsection*{3.\+2 Simulation with gazebo}

This package provides a simulation environment by employing package \char`\"{}rotors\+\_\+simulator\char`\"{}. There, the chaser is equipped with vision-\/sensor (vi-\/sensor) where you can check the actual capture of the target. In contrast to the \href{#without}{\tt {\itshape without gazebo} mode}, the initial spawn pose of the chaser should be set with the arguments {\bfseries chaser\+\_\+x} , {\bfseries chaser\+\_\+y}.


\begin{DoxyCode}
1 roslaunch auto\_chaser simulation\_with\_gazebo.launch
\end{DoxyCode}


The remaining procedure is the same with the section 3.\+1.

\section*{4. R\+OS \hyperlink{struct_node}{Node} A\+PI}

\subsection*{4.\+1 Published and subscribed topics}

\subsection*{4.\+2 Parameters}

\subsection*{}